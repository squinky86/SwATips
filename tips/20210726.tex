One project was marked for an operational buffer overflow (CWE-119) and poor coding practices when a buffer reading in a configuration file was set to a static size before reading in the file. The code sent for validation to fix the issue looked similar to that of Listing~\ref{lst:20210726:toctou}.

\begin{lstlisting}[caption={Overflows and TOCTOUs},captionpos=b,style=CStyle,basicstyle=\small,label={lst:20210726:toctou},numbers=left,stepnumber=1]
fseek(fp, 0, SEEK_END);
long size = ftell(fp);
char *buf = malloc(sizeof(char)*size);
fseek(fp, 0, SEEK_SET);
int c;
for (int i = 0; (c = getc(fp)) != EOF; i++)
{
	if (c == '%')
		break;
	buf[i] = c;
}
\end{lstlisting}

The developer's intent is to read a file into a buffer up to an expected truncation character, but the attempted fixes to the original issues caused more problems than the original code.

First, the code is subject to a Time-Of-Check Time-Of-Use (TOCTOU) issue. By setting up a file system watcher event, an attacker could append a pernicious payload to the end of the file so that the file is larger than \texttt{ftell()} originally reported. This causes a buffer overflow (CWE-120) on Line 10.

Second, the developer created a possible wrap-around issue (CWE-190) when the file's character count is larger than \texttt{INT\_MAX}. The software would then operate outside of the intended buffer boundary on Line 10 (CWE-119).

Finally, while inspecting the operational environment, it was discovered that the configuration file had world write privileges (CWE-276) which could be used to exploit the TOCTOU issue. Coincidentally, the file also had cleartext passwords (CWE-256) for a connected device.

What a mess a single TOCTOU issue can uncover!
