To combat password reuse, several tools have been devised such as KeePass, 1Password, and Bitwarden. Even most web browsers now, like Google Chrome, can manage pseudorandom passwords that are reasonably strong and unique. Each of these tools provides a way to generate cryptographically secure random passwords, but there is an important limitation to these passwords that makes them a little less secure than they could be:
\begin{quote}
The \textbf{pwgen} program generates passwords which are designed to be easily memorized by humans, while being as secure as possible. Human-memorable passwords are never going to be as secure as completely completely random passwords. In particular, passwords generated by \textbf{pwgen} without the \textbf{-s} option should not be used in places where the password could be attacked via an off-line brute-force attack.\autocite{20210809:pwgen}
\end{quote}

The goal of these password generation utilities is to generate passwords that can be written down, stored, and refernced. But are all passwords ones that fall into this category of pronouncability? Certainly not!

\subsection{Password Categories}

The majority of passwords are ones that may need to be communicated to someone. Passwords that are used for shared accounts and passwords that must be easily typed on a keyboard fall into this category. But there is another password category that we must consider: the temporary password.
