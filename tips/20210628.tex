Recently, a large software project encountered a serious vulnerability. The project has been in existence since the early 1990s and consists of millions of lines of code. On May 20\textsuperscript{th}, CISA issued an advisory that resulted in several CVEs being created for even some of the older RTOSes.\autocite{20210628:ics21-119-04} When we marked this as a critical finding against the legacy system, the response we received was, ``I didn't even know that was in there!''

\subsection{Establishing a Pedigree}

While the DoD is still proceeding with its RMF implementation of NIST 800-53rev4, Revision 5 of the RMF controls create a new family: SR - Supply Chain Risk Management. One of the new controls is SR-4, related to the provenance of the supply chain. To date, no categorization baseline requires the implementation of SR-4; however, it should be anticipated that this control will be tailored in for high-importance and tactical systems in the future. 

Part of SR-4 is enhancement SR-4(4) which requires the establishment of provenance and pedigree by keeping up with the internal composition of software and hardware components. ``For software this includes the composition of open-source and proprietary code, including the version of the component at a given point in time.''\autocite[\pno~]{20210628:nist80053rev5} This is a step above the hardware and software lists currently implemented in RMF; the program must manage the composition at a more granular level! Claiming not to know that a component includes a dependency would be a failure against this control.

\subsection{Compliance}

For software, I recommend implementing Software Package Data Exchange (SPDX) and a Software Bill of Materials (SBOM). A compliant policy would include a statement like the following:

\begin{quote}
Our organization requires that each main software delivery must define a Software Package Data Exchange (SPDX) file and Software Bill of Materials (SBOM) detailing all first-order dependencies.\autocite{20210628:wheeler2019} The SPDX file, at a minimum, must include the PackageName and PackageLicenseDeclared. If they are available, PackageOriginator and PackageHomePage must also be provided. Each main delivery product will also provide a SBOM consisting of an SPDX file for each first-order dependency (For example, if Product A is the deliverable and it depends on Product B, Product B is a first-order dependency. If Product B depends on Product C, Product C is a second-order dependency and only requires documentation in the SBOM if it is also a first-order dependency.).

For every major release of the deliverable or every three years (whichever occurs first), a Software Composition Analysis is performed, and the SBOM's SPDX files are compared to the results. The software composition analysis may be conducted automatically as part of the CI/CD pipeline or in our Software Assurance assessments. Undocumented dependencies are triaged as security concerns in our issue tracking system.
\end{quote}

Such a policy requires listings of dependencies with their version numbers, encourages developers to automate composition analysis, documents the POCs for each dependency, and records the license restrictions of each component.

\subsection{Recommendations}

The Application Sescurity and Development STIG should be updated to check that dependencies are documented appropriately. There is not currently a STIG requirement to document and manage dependencies.

When issuing CCIs against SA-4(4), the DoD should divide this into at least two checks: one for documenting the composition correctly, and the other for verifying the integrity of that composition.
